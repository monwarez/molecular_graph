\section{Le noyau de l'article : des marches al\'eatoires sur les graphes mol\'eculaires}
    \par
        On va dans la suite consid\'erer des graphes \'etiquett\'es, ce qui consiste \`a prendre un graphe
        $G = (V,E)$ et une fonction $l: V \cup E \rightarrow A$ o\`u $A$ est un alphabet choisi.
        Dans la suite l'alphabet $A$ est suppos\'e fix\'e (contenant les diff\'erents type d'atomes
        et de liaisons). On d\'efinit $V^{*}$ l'ensemble des suites finies de sommets.
        Au lieu de travailler avec des sous-graphes, nous allons \'etudier des chemins
        sur le graphe. Pour cel\`a on va identifier les suites d \'etiquettes comme des chemins sur le 
        graphe.
    
    \par
        Ici, les graphes sont parcourus selon une marche al\'eatoire, ainsi le noyau entre deux graphes
        est alors d\'efinit comme l'\'esperance d'avoir une paire de chemin similaire. Soit $p_{G1}$ ,
        ${p_{G2}}$ les densit\'ees de probabilit\'ees sur les ensembles de chemins $V_1^{*}$, resp. $V_2^{*}$.
        On a alors le noyau d\'efinit comme suit:
        \begin{equation}
            K(G_1,G_2) = \sum_{(h_1,h_2) \in V_1^{*} \times V_2^{*}} p_{G1}(h_1) p_{G2}(h_2) K_L(l(h_1),l(h_2))
        \end{equation}
        O\`u $l(h)$ est la succession des labels composant le chemin.

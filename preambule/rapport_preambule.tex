\usepackage[top=2cm,bottom=2cm,left=2cm,right=2cm]{geometry}% On peut changer en cours de route avec la commande \newgeometry

\usepackage[utf8]{inputenc}
\usepackage[english,french]{babel}
%\selectlanguage{english}
\usepackage{amsmath,amsfonts,amssymb,graphicx}
%\usepackage{amsthm}
\usepackage{framed}
\usepackage[amsthm,framed]{ntheorem}%%% option thmmarks apparemment incompatible avec \eqref
\usepackage[dvipsnames]{pstricks}
%\usepackage{pstricks-add,pst-plot,pst-node}
\usepackage{pstricks,pst-plot,pst-text,pst-tree}%,pst-eps,pst-fill,pst-node,pst-math,pstricks-add,pst-xkey}
\usepackage{epic,eepic}
% Versions propos\'ees par Frédéric Junier
\usepackage[inline]{asymptote}


%\usepackage{color} %(black, white, red, green, blue, yellow, magenta et cyan)
\usepackage[dvipsnames]{xcolor}
\usepackage{colortbl}

\usepackage{verbatim}

%\usepackage{enumitem}
%\frenchbsetup{StandardLists=true}

\usepackage[all]{xy}

\usepackage{multicol}				%%%	Jusqu' \`a nouvel ordre pas de multicol, pour cause de compatibilit\'e avec des fonctions graphiques de TeX.
%\begin{multicols}[titre]{nb colonnes}
%\setlength{\columnseprule}{0.25pt}

\usepackage{parskip}
\setlength{\parindent}{0cm}

%Je veux une mise en page pour les théorèmes lemmes preuves, définitions etc, suffisamment aérée. Est-il possible de commander un encadrement systématique ?

\newcommand{\g}[1]{\`#1}%\g{lettre}

\providecommand{\abs}[1]{\lvert#1\rvert}

%Pour tout le m\'emoire !

\newcommand{\der}[2][t]{\dfrac{\text{d}#2}{\text{d}#1}}

%Annotations d'un m\'emoire en cours de r\'edaction :
\newcommand{\tr}{\textbf{\textcolor{red}{D\'emonstration \g{a} trouver }}}
\newcommand{\re}{\textbf{\textcolor{NavyBlue}{D\'emonstration \g{a} recopier }}}
\newcommand{\es}{\textbf{\textcolor{OliveGreen}{Esquisse de d\'emonstration }}}

%Pour les formules alg\'ebriques casse-pieds :
\newcommand{\vect}[2]{\left(\begin{array}{c}#1\\#2\end{array}\right)}

%Pour d'autres formules spécifiques ... :


%Autres formules pour le m\'emoire :
\newcommand{\tc}[1]{\text{#1}}


\newcommand*{\etoile}
{
\begin{center}
\hspace{1pt}\par
*\hspace{5pt}*\hspace{5pt}*
\end{center}
}


\newcommand*{\ligneinter}
{
\begin{center}
\vspace{2pt}
\hfill\rule{0.5\linewidth}{0.1pt}\hfill\null
\end{center}
\vspace{7pt}
}

{
\theoremstyle{break}
\theoremprework{\vspace{0.2cm}\begin{minipage}{\textwidth}} %Pris en compte uniquement pour le premier newtheorem
\theorempostwork{\ligneinter\end{minipage}} %Même chose
\theoremheaderfont{\scshape}
\theorembodyfont{\normalfont}
\theoremseparator{ :\newline\vspace{0.2cm}}
\newtheorem{defi}{D\'efinition}%[section]
%\newtheorem{prop}{Proposition}[section]
%\newtheorem{lemm}{Lemme}[section]
} 

{%
\theoremstyle{break}
\theoremprework{\vspace{0.2cm}\begin{minipage}{\textwidth}} %Pris en compte uniquement pour le premier newtheorem
\theorempostwork{\ligneinter\end{minipage}} %Même chose
\theoremheaderfont{\bfseries}
\theorembodyfont{\itshape}
\theoremseparator{ :\newline\vspace{0.2cm}}
\newtheorem{prop}{Proposition}[section]
}

{%
\theoremstyle{break}
\theoremprework{\vspace{0.2cm}\begin{minipage}{\textwidth}} %Pris en compte uniquement pour le premier newtheorem
\theorempostwork{\ligneinter\end{minipage}} %Même chose
\theoremheaderfont{\bfseries}
\theorembodyfont{\itshape}
\theoremseparator{ :\newline\vspace{0.2cm}}
\newtheorem{pref}{Proposition-definition}[section]
}

{%
\theoremstyle{break}
\theoremprework{\vspace{0.2cm}\begin{minipage}{\textwidth}} %Pris en compte uniquement pour le premier newtheorem
\theorempostwork{\etoile\end{minipage}} %Même chose
\theoremheaderfont{\scshape}
\theorembodyfont{\itshape}
\theoremseparator{ :\newline\vspace{0.2cm}}
\newtheorem{conj}{Conjecture}
}

{%
\theoremstyle{break}
%\theoremprework{\begin{tabular}{|p{\textwidth}|}\hline}
%\theorempostwork{\\ \hline\end{tabular}}
\theoremheaderfont{\scshape}
\theorembodyfont{\normalfont}
\theoremseparator{ :\newline\vspace{0.2cm}}
%\newtheorem{lemm}{Lemme}[section]
\newframedtheorem{theo}{Theorem}[section]
}

{%
\theoremstyle{break}
\theoremprework{\vspace{0.2cm}\begin{minipage}{\textwidth}}
\theorempostwork{\ligneinter\end{minipage}}
\theoremheaderfont{\scshape}
\theorembodyfont{\itshape}
\theoremseparator{ :\newline\vspace{0.2cm}}
\newtheorem{lemm}{Lemma}[theo]
}
  
{%
\theoremstyle{break}
%\theoremprework{\begin{tabular}{|p{\textwidth}|}\hline}
%\theorempostwork{\\ \hline\end{tabular}}
\theoremheaderfont{\scshape}
\theorembodyfont{\itshape}
\theoremseparator{ :\newline\vspace{0.2cm}}
%\newtheorem{lemm}{Lemme}[section]
\newframedtheorem{coro}{Corollary}[theo]
}



{
\theoremstyle{break}
\theoremprework{\vspace{0.5cm}}
\theorempostwork{\vspace{0.5cm}\ligneinter}
\theoremheaderfont{\scshape}
\theorembodyfont{\normalfont\small}
\theoremseparator{ :\newline\vspace{0.2cm}}
\newtheorem{exem}{Example}[section]
} 

{
\theoremstyle{break}
\theoremprework{\vspace{0.5cm}\begin{minipage}{\textwidth}}
\theorempostwork{\end{minipage}}%\ligneinter}
\theoremheaderfont{\scshape}
\theorembodyfont{\small}
\theoremseparator{ :\newline\vspace{0.2cm}}
\newtheorem{rema}{Remarque}[section]
} 
  
%\setcounter{section}{-1}
\renewcommand{\thesection}{\Roman{section}}
\renewcommand{\thesubsection}{\Roman{section}-\Alph{subsection}}

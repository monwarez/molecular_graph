\section{Utilisation en pharmacologie}

\subsection{Noyaux marginalis\'es}

\begin{frame}
\begin{block}{Ind\'ependance conditionnelle sym\'etrique}<+->
On peut d\'efinir un noyau semid\'efini positif $K$ de fa\c con abstraite :
\[K(x,x')=\sum\limits_{\h}p(x|\h)p(x'|\h)p(\h)\]
o\g{u} $\h$ est une variable cach\'ee, parcourant un espace de probabilit\'e sur lequel l'\'int\'egrale $K$ est d\'efinie.
\end{block}
\begin{definition}<+->
\begin{itemize}
\item<+-> La formule, abstraite, qui pr\'ec\g{e}de peut \^etre r\'eecrite : %
\[K(x,x')=\sum\limits_{h,h'}K_z(z,z')p(x|h)p(x'|h')\]
\item<+-> Cette approche Bayesienne pr\'esente l'avantage que les distributions a posteriori $p(\cdot|\h)$ peuvent \^etre calcul\'ees.
\item<+-> $K_z$ est appel\'e la \emph{noyau conjoint}, qui prend en compte les informations $x$, $x'$ et les variables cach\'ees.
\end{itemize}
\end{definition}
\end{frame}

\subsection{Cas des graphes \'etiquet\'es : avec des marches al\'eatoires}

\begin{frame}
\begin{definition}<+->
Un \textbf{graphe \'etiquet\'e} $G$ est d\'edini par le couple $(V,E)$ de ses sommets et ar\g{e}tes, et une application $l$ de $V\cup E$ vers un alphabet $\mathcal{A}$.
\end{definition}
\begin{block}{Chimie organique}<+->
\begin{itemize}
\item<+-> Les mol\'ecules construites avec des liaisons covalentes, notamment en chimie organique, peuvent \g{e}tre mod\'elis\'ees par des graphes connexes \'etiquet\'es.
\item<+-> Dans ce cas, les \'etiquettes de sommets correspondent aux diff\'erents atomes, celles des ar\g{e}tes aux diff\'erents types de liaisons.
\end{itemize}
\end{block}
\end{frame}

\begin{frame}
\begin{block}{}<+->
L'approche pr\'econis\'ee ici est de prendre pour $\h$, $\h'$ des marches al\'eatoires sur les graphes mol\'eculaires. %
On extrait les informations $x$, $x'$ des marches images contenues \emph{dans $\mathcal{A}^{\ast}$} pour alimenterle noyau conjoint.
\end{block}
\begin{block}{Noyau marginalis\'e}<+->
\[K(G_1,G_2) = \sum\limits_{(h_1,h_2) \in V_1^{\ast} \times V_2^{\ast}}p(\h_1|G_1)p(\h_2|G_2)K_L(l(\h_1),l(\h_2))\]
\end{block}
\end{frame}

\begin{frame}
\begin{itemize}
\item<+-> $K_L$ et les distributions $p(\cdot |\h)$ sont calcul\'ees inductivement, en prenant en compte le caract\g{e}re markovien des marches $\h$ ;
\item<+-> $K_L$ porte sur des marches potentiellement infiniment longues ;
\item<+-> la convergence du noyau conjoint est assur\'ee et on peut l'approcher moyennant la r\'esolution de syst\g{e}mes lin\'eaires.
\end{itemize}
\end{frame}

\subsection{Quelques raffinements}

\begin{frame}
\begin{block}{Indice de Morgan}<+->
\begin{itemize}
\item<+-> D\'efini inductivement sur les sommets des graphes mol\'eculaires, initialis\'e \g{a} $1$ partout
\item<+-> A chaque \'etape, chaque sommet se voit affecter la somme des indices de ses plus proches voisins
\item<+-> Calculable simplement par it\'erations successives de la matrice d'adjacence du graphe
\end{itemize}
\end{block}
\begin{block}{Pr\'evention du b\'egaiement}<+->
On peut \'eviter de g\'en\'erer des marches al\'eatoires dites b\'egayantes en simulant un processus markovien \g{a} plusieurs ant\'ec\'edents.
\end{block}
\end{frame}

% Ne pas mentionner les fragments descripteurs sur le diaporama, seulement oralement
% Graphes \'etiquet\'es : d\'efinitions, cas de la chimie organique
